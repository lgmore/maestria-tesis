\documentclass[a4,11pt]{report}

\usepackage[brazil]{babel}      % para texto em Portugu�s
%\usepackage[english]{babel}    % para texto em Ingl�s

\usepackage[latin1]{inputenc}   % para acentua��o em Portugu�s OU
%\usepackage[utf8]{inputenc}   % para acentua��o em Portugu�s com o uso do Unicode, 
% mude a codifica��o desse template para utf-8

%%
%% POR FAVOR, N�O FA�A MUDAN�AS NESSE TEMPLATE QUE ACARRETEM  EM
%% ALTERA��O NA FORMATA��O FINAL DO TEXTO
%%
\usepackage{graphics}
\usepackage{subfigure}
\usepackage{graphicx}
\usepackage{epsfig}
\usepackage[centertags]{amsmath}
\usepackage{graphicx,indentfirst,amsmath,amsfonts,amssymb,amsthm,newlfont}
\usepackage{longtable}
\usepackage{cite}
\usepackage[usenames,dvipsnames]{color}
\usepackage{natbib,har2nat}
\nocite{*}


\begin{document}

%********************************************************
\title{Multi-Objective Optimization for Image Denoising}

\author{
 {\large  Luis G. Mor�}\thanks{lmore@pol.una.py}\\
 {\small Facultad Polit�cnica - Universidad Nacional de Asunci�n} \\
{\large Mateus Bernardes}\thanks{mbernardes@uftpr.edu.br} \\
{\small Departamento Acad�mico de Matem�tica, UTFPR, Curitiba, PR} \\
 {\large Igor Leite Freire}\thanks{igor.freire@ufabc.edu.br}\\
 {\small Centro de Matem�tica, Computa��o e Cogni��o, UFABC}  
 }

\criartitulo


\section{ Introduction}

Image denoising is the proccess which consists in removing noise from digital images. There are several factors that make images susceptible to noise. It is possible to model a degradation function and a noise term $\eta(x,y)$ that operates over an input image $f(x,y)$ and a degraded image is obtained as a result \cite{Gonzalez:2003:DIP:993475}:

\begin{equation}
g(x,y) = H[f(x,y)] + \eta(x,y)
\label{Calor}
\end{equation}

Based on $g(x,y)$ it is necessary to obtain an estimate $f'(x,y)$ of the original image. 

A well-known technique for image denoising is Non-Local Means denoising algorithm \cite{buades2011non}, which is combined with Multi-Objective optimization, in order to obtain series of images with different compromise rates of denoising. Denoised images are evaluated using well-known metrics for denoising, which make this implementation suitable. 


\section{Preliminary Results}



\section{Preliminary Conclusions}
Em linhas gerais, as principais conclus�es do trabalho, se poss�vel (� facultativo!). 

\section*{Agradecimentos}
Apresentar os agradecimentos �s pessoas e institui��es pertinentes, se houver espa�o (� facultativo!).
\bibliographystyle{plain}
\bibliography{referencias} 
%  \begin{thebibliography}{referencias}

% % % \bibitem{Boldrini} J. L. Boldrini, S. I. R. Costa, V. R. Ribeiro, and H. G. Wetzler. {\it �lgebra Linear 
% % % e Aplica��es}. Harper-Row, S�o Paulo, 1987.

% % % \bibitem{Cuminato}
% % % J. A. Cuminato and V. Ruas, Unification of distance inequalities for linear variational problems, 
% % % {\it Comp. Appl. Math.}, 2014. DOI: 10.1007/s40314-014-0163-6.

% % % \bibitem{Diniz1}
% % % G. L. Diniz, A mudan�a no habitat de popula��es de peixes: de rio a represa -- o modelo matem�tico, Disserta��o de Mestrado em Matem�tica Aplicada, Unicamp, (1994).

% % % \bibitem{Jafelice} R. M. Jafelice, L. C. Barros and R. C. Bassanezi. Study of the dynamics of HIV under 
% % % treatment considering fuzzy delay, {\it Comp. Appl. Math.}, 33:45--61, 2014.

% % % \bibitem{Santos} I. L. D. Santos e G. N. Silva. Uma classe de problemas de controle �timo 
% % % em escalas temporais, {\it Proceeding Series of the Brazilian Society of Computational and 
% % % Applied Mathematics}, volume 1, 2013. DOI: 10.5540/03.2013.001.01.0177.

%  \end{thebibliography}
\end{document}




