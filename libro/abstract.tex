
\begin{foreignabstract}
Contrast Enhancement is a transformation function applied over a digital image, with the aim to obtain another image whose characteristics of contrast are more suitable for further image proccessing steps. There are several techniques for Contrast Enhancement of Digital Images, among them stand out the techniques of Contrast Enhancement based on Methaheuristics; those are well proven methods for grayscale images. The main objective is to obtain parameters for a constrast enhancement algorithm which are suitable for a digital image, which contrast problem is being addressed. Nevertheless, new difficulties arise when working with colored digital images, in the context of Contrast Enhancement based in pure Metaheuristics: not only is neccesary to achieve better contrast of one or more object in regard of the background, but also is neccesary to consider color information, which is also affected. 

This work addresses the problem of Contrast Enhancement of color images based in an pure Multiobjective approach. The proposal applies a well-known Metaheuristic to the input parameters of a Contrast Enhancement Algorithm, which results in images potentially suitable as solutions of the problem. Those are evaluated taking into account balance between contrast achieved and distortion of information whithin images (in terms of intensity and color information). The results obtained show images with better contrast, and non-dominated metric coefficients that show an inverse relation between contrast and structural similarity (distortion).
% Contrast enhancement is used as a preprocessing of other algorithms such as image segmentation, image fusion, among others. Contrast enhancement is of utmost importance, since a low-contrast image would cause these algorithms to present undesired results. The low contrast of the images may be due to several factors, such as poor lighting or faults with the acquisition medium. The problem of low contrast is solved using a technique that enhances the visual quality of the image. Mathematical morphology is one of the techniques that improves images with low contrast, and has demonstrated efficiency in improving the quality of grayscale images. For its application in color images, it is necessary to adopt a color space and to determine an order for the components of the vectors of the color image. Applications of different areas, such as medical sciences, engineering and geosciences, use contrast enhancement in preprocessing stages.

% This work presents an algorithm that improves the visual quality of grayscale images and color images. The proposed algorithm extracts image characteristics in multiple scales using mathematical morphology. The validation of the proposal was done using 200 color images from a public database. The size of the color images are $481 \times 321$ and $321 \times 481$. The comparison was performed with algorithms that modify the histogram and another that uses the multiscale top-hat transform. The evaluation of the experimental results was done with metrics that evaluate the local and global contrast. The proposed algorithm obtained better numerical and visual evaluations for all cases tested, both for grayscale images and color images.
%\textcolor{Micolor1}{Experimental results show that the proposed algorithm obtained better numerical and visual results for all tested cases, both for grayscale images and color images.} 
%\textcolor{Micolor2}{The proposed algorithm obtained better numerical and visual evaluations for all cases tested, both for grayscale images and color images.} 


\end{foreignabstract}

