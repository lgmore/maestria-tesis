\PassOptionsToPackage{spanish,es-noquoting}{babel}
\documentclass[final,fmstyle,unknownkeysallowed]{./util/fpunathesis}

% paquetes recomendados
\usepackage{amsmath,amsthm}
\usepackage{textcomp}
\usepackage[T1]{fontenc}
\usepackage[utf8]{inputenc}
\usepackage{csquotes}
\usepackage{enumerate}
\usepackage{enumitem}
\usepackage{caption}
\usepackage{subcaption}
\usepackage[style=alphabetic, uniquename=full, sorting=none,backend=biber, natbib=true]{biblatex}
\usepackage{listings}
% para la lista de simbolos
\usepackage{array} %for vertical thick lines in tables
\usepackage{multirow} %multirow tables
\usepackage{nicefrac} %for fractions like 1/4
\usepackage{pgfplots}
\pgfplotsset{compat=default}
\usepackage{pgfplotstable}
\usepgfplotslibrary{statistics}
\usepackage{tikz}
\usetikzlibrary{spy}
\usetikzlibrary{arrows}
\usepackage{longtable}
\usepackage{booktabs, colortbl}
\usepackage{array}
\usepackage{titlesec}
\usepackage{tabu}
\usepackage{pdflscape}
\usepackage[section, above]{placeins}
\usepackage{float}
\usepackage{acronym}
%referencias
\addbibresource{referencias.bib}
%se importan las configuraciones custmizdas realizadas.
\include{./util/custom}

% datos de la tesis y el/los autor/es
\title{Mejora de contraste utilizando morfología matemática multiescala para imágenes en escala de grises e imágenes en color}
\author{Julio César Mello Román}
\degree{Ciencias de la Computación}

\advisor{Dr. Horacio Legal Ayala  \\M.Sc. José Luis Vázquez Noguera}

%\newtheorem{definicion}{Definición}

\logosource{./Figures/fpuna.jpg}


\begin{document}
	
%%\tikzset{mark options={mark size=1.5, fill}} %para que las marcas no sean tan grandes
%\makeatletter
%\renewcommand\paragraph{\@startsection{paragraph}{4}{\z@}%
%	{-2.5ex\@plus -1ex \@minus -.25ex}%
%	{1.25ex \@plus .25ex}%
%	{\normalfont\normalsize\bfseries}}
%\makeatother
%\setcounter{secnumdepth}{4} % how many sectioning levels to assign numbers to
%\setcounter{tocdepth}{4}  

\maketitle     % esto hace las portadas

% Dedicatoria
\chapter*{}
%\pagenumbering{Roman} % para comenzar la numeracion de paginas en numeros romanos
\begin{flushright}
	\textit{Dedicado a mi esposa Maura Soledad y a mis hijos Julio, Maira y Melisa. \\Ustedes son el motor que me empuja hacia adelante.\\ \textbf{Los amo.}\\
	Dedicado a mis padres Lino y Mar\'ia Egberta Ester. \\Gracias por brindarme en esta vida su apoyo incondicional.\\
	\textbf{Julio C\'esar}}
\end{flushright}



% Agradecimientos
\chapter*{Agradecimientos}

Quisiera agradecer en primer lugar a mi familia, quienes me dan el soporte necesario para seguir adelante. También agradezco a mis mentores, el Prof. Dr. Diego Pinto, y el MSc. José Vázquez, su ayuda fué invaluable en este trabajo.

Agradezco al NIDTEC por brindarme la oportunidad de realizar éste curso de postgrado.

Agradezco al CONACYT por la oportunidad de incentivo al programa de Maestría al que me tocó acceder.

Y más que nada, agradezco a mi Padre y mi Madre Divinos, quienes en secreto me cuidan.


% Resumen
\begin{abstract}
[INSERTE ABSTRACT AQUI]
% La mejora del contraste se utiliza como un preprocesamiento de otros algoritmos como la segmentación de imágenes, fusión de imágenes, entre otros. La mejora del contraste es de suma importancia, ya que una imagen con bajo contraste haría que estos algoritmos arrojen resultados indeseados. El bajo contraste de las imágenes puede darse por varios factores, como la iluminación deficiente o fallas con el medio de adquisición. El problema del bajo contraste se soluciona utilizando una técnica que realza la calidad visual de la imagen. La morfología matemática es una de las técnicas que mejora las imágenes con bajo contraste, y ha demostrado eficiencia en la mejora de la calidad de las imágenes en escala de grises. Para su aplicación en imágenes en color es necesario adoptar un espacio de color y determinar un orden para los componentes de los vectores de la imagen en color. Aplicaciones de diferentes áreas, como las ciencias médicas, ingenierías y geociencias, aplican la mejora del contraste en etapas de preprocesamiento. 

% En este trabajo se presenta un algoritmo que mejora la calidad visual de imágenes en escala de grises e imágenes en color. El algoritmo propuesto extrae características de la imagen en escalas múltiples utilizando la morfología matemática. La validación de la propuesta se realizó utilizando 200 imágenes en color de una base de datos pública. El tamaño de las imágenes en color son de $481 \times 321$ y de $321 \times 481$. La comparación se realizó con algoritmos que modifican el histograma y otra que utiliza la transformada de top-hat multiescala. La valoración de los resultados experimentales se realizaron con métricas que evalúan el contraste local y global. El algoritmo propuesto obtuvo mejores valoraciones numéricas y visuales para todos los casos probados, tanto para imágenes en escala de grises e imágenes en color.
%\textcolor{Micolor1}{Los resultados experimentales muestran que el algoritmo propuesto obtuvo mejores resultados numéricos y visuales para todos los casos probados, tanto para imágenes en escala de grises e imágenes en color.} 
%\textcolor{Micolor2}{El algoritmo propuesto obtuvo mejores valoraciones numéricas y visuales para todos los casos probados, tanto para imágenes en escala de grises e imágenes en color.} 
\end{abstract}



% Tabla de contenidos
\tableofcontents
% Lista de figuras, incluye en la lista todas las figuras de forma automática
\listoffigures
% Lista de tablas, incluye en la lista todas las tablas de forma automática
\listoftables
% Lista de algoritmos, incluye en la lista todas los algoritmos de forma automática
%\listofalgorithms
%\include{acronimos}

\chapter*{LISTA DE ABREVIATURAS}
%\myheader{LISTA DE ABREVIATURAS}
\markboth{LISTA DE ABREVIATURAS}{LISTA DE ABREVIATURAS}
\addcontentsline{toc}{chapter}{LISTA DE ABREVIATURAS}
%Indice de abreviaturas
\acrodef{RGB}{RGB}
RGB: Espacio de color RGB.

\acrodef{HSI}{HSI}
HSI: Espacio de color HSI.

\acrodef{HSV}{HSV}
HSV: Espacio de color HSV.

\acrodef{HE}{HE}
HE: \textit{Histogram Equalization}.

\acrodef{CLAHE}{CLAHE}
CLAHE: \textit{Contrast-Limited Adaptive Histogram Equalization}.

\acrodef{MMCE}{MMCE}
MMCE: \textit{Multiscale Morphological Contrast Enhancement}.

\acrodef{C}{C}
C: \textit{Contrast}.

\acrodef{CIR}{CIR}
CIR: \textit{Contrast Improvement Ratio}.

\acrodef{CEF}{CEF}
CEF: \textit{Color Enhancement Factor}.

\listofsymbols

\mainmatter  % inician los capitulos de la tesis

\include{util/read_color_mappings} %carga configuración de color por filtro y algunos comandos

% incluye aqui los capítulos (un archivo .tex por capítulo)
\include{capitulo-1/capitulo-1}
\include{capitulo-2/capitulo-2}
\include{capitulo-3/capitulo-3}
\include{capitulo-4/capitulo-4}
\include{capitulo-5/capitulo-5}
%\include{capitulo-6/capitulo-6}

\appendix   % inician los apendices de tu tesis
% los capítulos que incluyas a partir de aquí aparecen
% como apéndices

%\include{anexos/anexos-2}
% estos comandos generan la bilbiografía
\printbibliography

\include{anexos/anexos}

\end{document}