\chapter{INTRODUCCIÓN}

La Mejora del Contraste es un paso de preprocesamiento fundamental para varias aplicaciones tales como las Imágenes Médicas (como ejemplos es posible tomar: el Diagnóstico Asistido por Computadora \cite{doi2007computer}, Imágenes de Tomografía Computarizada \cite{doi:10.1056/NEJM199303113281008}, y otros), Sensoreamiento Remoto \cite{lillesand2014remote}, y otros.

Las técnicas basadas en Ecualización del Histograma se mostraron extensivamente válidas para enfocar los problemas de Mejora del Contraste \cite{Gonzalez02a,pizer1987adaptive,zuiderveld1994contrast,580378}. Las Meta-Heurísticas tales como la Optimización Mono-Objetivo, y también la Optimización Multi-Objetivo fueron testeadas satisfactoriamente de manera a resolver problemas de Mejora del Contraste en imágenes en escala de gris \cite{morepso,more2015parameter,812529,HOSEINI2013879}. Sin embargo, la Optimización Multi-Objetivo aplicada a la Mejora del Contraste en imágenes a color supone dificultades adicionales, debido a que es necesario preservar la información de color presente dentro de dichas imágenes.

Ésta propuesta consiste en realizar pruebas con imágenes a color transformadas desde el espacio de colores $RGB$ al espacio de colores $YCbCr$ de manera a realizar la Mejora de Contraste basada en Optimización Multi-Objetivo. Contrast Limited Adaptive Histogram Equalization (CLAHE) se aplica sobre el canal $Y$ de la imagen de prueba, de manera a modificar el contraste, y la imagen resultante se transforma nuevamente a $RGB$ de forma a evaluar la similaridad entre canales de color.

%Our proposal consist in testing images transformed from $RGB$ color space to $YCbCr$ in order to perform MMO-based CE. Contrast Limited Adaptive Histogram Equialization (CLAHE) is applied over the $Y$ channel of the test image in order to modify contrast, and the resultant image is transformed back to $RGB$ in order to evaluate the similarity between color channels.

%The rest of the paper is organized as follows: in Section \ref{sec:theorethical_framework}, the fundametal concepts for this work are presented, in Section \ref{sec:proposal} the CE problem is posed, and our approach is presented, in Section \ref{sec:results_discussion} the results achieved are discussed in detail, and finally in \ref{sec:conclusion} some final points are remarked.


\section{Objetivos}
\subsection{Objetivo General}
Desarrollar un algoritmo de mejora de contraste para imágenes a color, utilizando un enfoque de Metaheurística Multi-Objetiva pura. 
%Desarrollar un algoritmo de mejora de contraste para imágenes en escala de grises e imágenes en color que utiliza la matemática morfológica multiescala.
\subsection{Objetivos específicos}
\begin{itemize}

	\item Desarrollar un nuevo algoritmo de Mejora del Contraste de imágenes a color basado en Metaheurísticas Multi-Objetivo.

	\item Demostrar la factibilidad del enfoque de Mejora de Contraste de imágenes a color basado en Metaheurísticas Multi-Objetivo puras.

	\item Encontrar alternativas de implementación que ayuden a subsanar problemas inherentes a los enfoques basados en Metaheurísticas Multi-Objetivo, cuando la cantidad de objetivos sobrepasa a tres.
% 	\item Proponer un nuevo algoritmo que utiliza matemática morfológica multiescala para la mejora del contraste de imágenes en escala de grises e imágenes en color.
	
% 	\item Comparar el algoritmo propuesto con algoritmos que modifican el histograma, tanto local como global, en imágenes en escala de grises.
	
% 	\item Comparar el algoritmo propuesto con un algoritmo que utiliza la transformada de top-hat en escalas múltiples en imágenes en escala de grises.
	
% 	\item Establecer relaciones de orden en los espacios de color RGB, HSI y HSV, de tal forma que el algoritmo propuesto sea aplicable a imágenes en color.
	
% 	\item Comparar el algoritmo propuesto con algoritmos que modifican el histograma, tanto local como global, en imágenes en color.
	
% 	\item Comparar el algoritmo propuesto con un algoritmo que utiliza la transformada de top-hat en escalas múltiples en imágenes en color.
	
% 	\item Comparar los tiempos de computo del algoritmo propuesto con un algoritmo que utiliza la transformada de top-hat en escalas múltiples.
	
\end{itemize}

\section{Estructura de la tesis}
El trabajo, en las secciones siguientes se organiza de la siguiente manera: en el capítulo \ref{sec:theorethical_framework}, los conceptos fundamentales de éste trabajo se presentan; en el capítulo \ref{sec:proposal} se presenta el problema de Mejora de Contraste, y el enfoque de éste trabajo se muestra; en el capítulo \ref{sec:results_discussion} se discute en detalle los resultados obtenidos, y finalmente en el capítulo \ref{sec:conclusion} se hacen algunos comentarios finales.
