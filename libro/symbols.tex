\chapter*{LISTA DE S\'IMBOLOS}
%\myheader{LISTA DE S\'IMBOLOS}
\markboth{LISTA DE S\'IMBOLOS}{LISTA DE S\'IMBOLOS}
\addcontentsline{toc}{chapter}{LISTA DE S\'IMBOLOS}

\begin{tabular}{rl}
$I$							& \parbox{4.6in}{Imagen original\dotfill \pageref{symbol:ioriginal}} \\
$T$							& \parbox{4.6in}{Imagen con contraste mejorado\dotfill \pageref{symbol:imejorada}} \\
$I_y$							& \parbox{4.6in}{Canal $Y$ del espacio $YCbCr$ de la imagen original\dotfill \pageref{symbol:ioriginaly}} \\
$T_y$							& \parbox{4.6in}{Canal $Y$ del espacio $YCbCr$ de la imagen contrastada\dotfill \pageref{symbol:imejoraday}} \\
$I_R$							& \parbox{4.6in}{Canal $R$ del espacio $RGB$ de la imagen original\dotfill \pageref{symbol:ioriginalr}} \\
$T_R$							& \parbox{4.6in}{Canal $R$ del espacio $RGB$ de la imagen contrastada\dotfill \pageref{symbol:imejoradar}} \\
$I_G$							& \parbox{4.6in}{Canal $G$ del espacio $RGB$ de la imagen original\dotfill \pageref{symbol:ioriginalg}} \\
$T_G$							& \parbox{4.6in}{Canal $G$ del espacio $RGB$ de la imagen contrastada\dotfill \pageref{symbol:imejoradag}} \\
$I_B$							& \parbox{4.6in}{Canal $B$ del espacio $RGB$ de la imagen original\dotfill \pageref{symbol:ioriginalb}} \\
$T_B$							& \parbox{4.6in}{Canal $B$ del espacio $RGB$ de la imagen contrastada\dotfill \pageref{symbol:imejoradab}} \\
\ensuremath{\mathscr{H}}		& \parbox{4.6in}{Entropía de la imagen digital\dotfill \pageref{symbol:entropia}} \\ 

$SSIM$							& \parbox{4.6in}{Índice de Similaridad Estructural\dotfill \pageref{symbol:ssim}} \\
$\vv{x}$							& \parbox{4.6in}{Partícula componente de \textit{MOPSO}\dotfill \pageref{symbol:mopsoparticula}} \\
$\vv{v}$							& \parbox{4.6in}{Componente de velocidad de \textit{MOPSO}\dotfill \pageref{symbol:mopsovelocidad}} \\
$\mathscr{X}$							& \parbox{4.6in}{Conjunto de soluciones no dominadas del algoritmo $CMOPSO-CLAHE$\dotfill \pageref{symbol:nodominadas}} \\
$\mathscr{R}_x$						& \parbox{4.6in}{Parámetro de ventana $x$ de \textit{CLAHE}\dotfill \pageref{symbol:claheventanax}} \\
$\mathscr{R}_y$							& \parbox{4.6in}{Parámetro de ventana $y$ de \textit{CLAHE}\dotfill \pageref{symbol:claheventanay}} \\
$h_{f_{k}}(j)$				& \parbox{4.6in}{Histograma del canal $f_k$\dotfill \pageref{symbol:hfk}} \\
$n_{j}$						& \parbox{4.6in}{Cantidad de ocurrencia de la intensidad $j$ en $f_{k}$\dotfill \pageref{symbol:nj}} \\
$g$							& \parbox{4.6in}{Elemento estructurante\dotfill \pageref{symbol:se}} \\
(u,v)						& \parbox{4.6in}{Coordenada espacial que representa un pixel de la imagen\dotfill \pageref{symbol:uv}} \\
(s,t)						& \parbox{4.6in}{Coordenada espacial del elemento estructurante\dotfill \pageref{symbol:st}} \\
($f \oplus g$)				& \parbox{4.6in}{Dilatación de la imagen original $f$ por un elemento estructurante $g$\dotfill \pageref{symbol:dil}} \\
($f \ominus g$)				& \parbox{4.6in}{Erosión de la imagen original $f$ por un elemento estructurante $g$\dotfill \pageref{symbol:ero}} \\
($f \circ g$)				& \parbox{4.6in}{Apertura de la imagen original $f$ por un elemento estructurante $g$\dotfill \pageref{symbol:ape}} \\
($f \bullet g$)				& \parbox{4.6in}{Cierre de la imagen original $f$ por un elemento estructurante $g$\dotfill \pageref{symbol:cie}} \\
$WTH$						& \parbox{4.6in}{Transformada de top-hat por apertura\dotfill \pageref{symbol:wth}} \\
$BTH$						& \parbox{4.6in}{Transformada de top-hat por cierre\dotfill \pageref{symbol:bth}} \\
$f_{E}$						& \parbox{4.6in}{Imagen con mejora de contraste\dotfill \pageref{symbol:fen}}
\end{tabular}


\newpage
\begin{tabular}{rl}

	$WTH_{i}$					& \parbox{4.6in}{$i$-escalas de brillos\dotfill \pageref{symbol:wthi}} \\
	$BTH_{i}$					& \parbox{4.6in}{$i$-escalas de oscuridad\dotfill \pageref{symbol:bthi}} \\
	$WTH^{S}_{i-1}$				& \parbox{4.6in}{$(i-1)$-diferencias en cascada de las escalas de brillo\dotfill \pageref{symbol:wthsi}} \\
	$BTH^{S}_{i-1}$				& \parbox{4.6in}{$(i-1)$-diferencias en cascada de las escalas de oscuridad\dotfill \pageref{symbol:wthsi}} \\
	$WTH_{M}$					& \parbox{4.6in}{Valores máximos de todas las escalas de brillos\dotfill \pageref{symbol:wthm}} \\
	$BTH_{M}$					& \parbox{4.6in}{Valores máximos de todas las escalas de oscuridad\dotfill \pageref{symbol:bthm}} \\
	$WTH^{S}_{M}$				& \parbox{4.6in}{Valores máximos de todas las escalas de brillos por sustracción\dotfill \pageref{se_03_p}} \\
	$BTH^{S}_{M}$				& \parbox{4.6in}{Valores máximos de todas las escalas de oscuridad por sustracción\dotfill \pageref{se_04_p}} \\
	$E(f)$						& \parbox{4.6in}{Intensidad media de la imagen $f$\dotfill \pageref{symbol:E}} \\
	$P(j)$						& \parbox{4.6in}{Probabilidad de ocurrencia del valor $j$\dotfill \pageref{symbol:P}} \\
	$\rho$						& \parbox{4.6in}{Valor del pixel central dentro de una ventana\dotfill \pageref{rho}} \\
	$\iota$						& \parbox{4.6in}{Valor medio de los vecinos de $\rho$\dotfill \pageref{iota}} \\
	$\omega$					& \parbox{4.6in}{Contraste local\dotfill \pageref{lc}} \\
	$D$							& \parbox{4.6in}{Dominio de una imagen\dotfill \pageref{D}} \\
	$\gamma$					& \parbox{4.6in}{Diferencia entre los canales $f_1$ y $f_2$ de una imagen\dotfill \pageref{symbol:gamma}} \\
	$\beta$						& \parbox{4.6in}{Diferencia entre un medio de $(f_1+f_2)$ y $f_3$\dotfill \pageref{symbol:beta}} \\
	$\sigma_{\gamma}$			& \parbox{4.6in}{Desviación estándar de $\gamma$\dotfill \pageref{symbol:cm}} \\
	$\sigma_{\beta}$			& \parbox{4.6in}{Desviación estándar de $\beta$\dotfill \pageref{symbol:cm}} \\
	$\mu_{\gamma}$				& \parbox{4.6in}{Media aritmética de $\gamma$\dotfill \pageref{symbol:cm}} \\
	$\mu_{\beta}$				& \parbox{4.6in}{Media aritmética de $\beta$\dotfill \pageref{symbol:cm}} 
\end{tabular}