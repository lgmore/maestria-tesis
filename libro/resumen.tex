\begin{abstract}
La mejora del contraste se utiliza como un preprocesamiento de otros algoritmos como la segmentación de imágenes, fusión de imágenes, entre otros. La mejora del contraste es de suma importancia, ya que una imagen con bajo contraste haría que estos algoritmos arrojen resultados indeseados. El bajo contraste de las imágenes puede darse por varios factores, como la iluminación deficiente o fallas con el medio de adquisición. El problema del bajo contraste se soluciona utilizando una técnica que realza la calidad visual de la imagen. La morfología matemática es una de las técnicas que mejora las imágenes con bajo contraste, y ha demostrado eficiencia en la mejora de la calidad de las imágenes en escala de grises. Para su aplicación en imágenes en color es necesario adoptar un espacio de color y determinar un orden para los componentes de los vectores de la imagen en color. Aplicaciones de diferentes áreas, como las ciencias médicas, ingenierías y geociencias, aplican la mejora del contraste en etapas de preprocesamiento. 

En este trabajo se presenta un algoritmo que mejora la calidad visual de imágenes en escala de grises e imágenes en color. El algoritmo propuesto extrae características de la imagen en escalas múltiples utilizando la morfología matemática. La validación de la propuesta se realizó utilizando 200 imágenes en color de una base de datos pública. El tamaño de las imágenes en color son de $481 \times 321$ y de $321 \times 481$. La comparación se realizó con algoritmos que modifican el histograma y otra que utiliza la transformada de top-hat multiescala. La valoración de los resultados experimentales se realizaron con métricas que evalúan el contraste local y global. El algoritmo propuesto obtuvo mejores valoraciones numéricas y visuales para todos los casos probados, tanto para imágenes en escala de grises e imágenes en color.
%\textcolor{Micolor1}{Los resultados experimentales muestran que el algoritmo propuesto obtuvo mejores resultados numéricos y visuales para todos los casos probados, tanto para imágenes en escala de grises e imágenes en color.} 
%\textcolor{Micolor2}{El algoritmo propuesto obtuvo mejores valoraciones numéricas y visuales para todos los casos probados, tanto para imágenes en escala de grises e imágenes en color.} 
\end{abstract}

