\begin{abstract}
La mejora del contraste es una función de transformación aplicada a una imagen digital cuya finalidad es la de obtener una imagen cuyas características de contraste sean más adecuadas para una aplicación posterior de procesamiento. Existen diversas técnicas de Mejora del contraste de imágenes, de entre las que resaltan las técnicas basadas en enfoques Metaheurísticos; los mismos fueron probados extensivamente en la literatura, para imágenes en escala de grises. La finalidad es la de obtener parámetros de un algoritmo de mejora del contraste que sean adecuados para la imagen digital cuyo problema de mejora del contraste se está abordando. Sin embargo, aparecen nuevas dificultades cuando se trabaja con imágenes digitales a color, en el contexto de la Mejora del Contraste basada en Metaheurísticas puras: no solamente es necesario mejorar el contraste de uno o más objetos con respecto al fondo, sino que además es necesario considerar la información de color que también se ve afectada.
% La mejora del contraste se utiliza como un preprocesamiento de otros algoritmos como la segmentación de imágenes, fusión de imágenes, entre otros. La mejora del contraste es de suma importancia, ya que una imagen con bajo contraste haría que estos algoritmos arrojen resultados indeseados. El bajo contraste de las imágenes puede darse por varios factores, como la iluminación deficiente o fallas con el medio de adquisición. El problema del bajo contraste se soluciona utilizando una técnica que realza la calidad visual de la imagen. La morfología matemática es una de las técnicas que mejora las imágenes con bajo contraste, y ha demostrado eficiencia en la mejora de la calidad de las imágenes en escala de grises. Para su aplicación en imágenes en color es necesario adoptar un espacio de color y determinar un orden para los componentes de los vectores de la imagen en color. Aplicaciones de diferentes áreas, como las ciencias médicas, ingenierías y geociencias, aplican la mejora del contraste en etapas de preprocesamiento. 

Éste trabajo aborda el problema de Mejora del Contraste en imágenes a color con un enfoque multiobjetivo puro. El algoritmo propuesto aplica una Metaheurística bien conocida a los parámetros de un algoritmo de mejora del contraste, lo cual resulta en imágenes potencialmente adecuadas para ser consideradas como soluciones. Éstas se evaluan teniendo en cuenta el balance entre contraste obtenido y distorsión de la información contenida dentro de la imágenes (en términos de intensidad y de información de color). Los resultados obtenidos muestras imágenes con el contraste mejorado, pero cuyos coeficientes de métrica no dominados muestran una relación inversa de compromiso entre contraste y similaridad estructural (distorsión).
% En este trabajo se presenta un algoritmo que mejora la calidad visual de imágenes en escala de grises e imágenes en color. El algoritmo propuesto extrae características de la imagen en escalas múltiples utilizando la morfología matemática. La validación de la propuesta se realizó utilizando 200 imágenes en color de una base de datos pública. El tamaño de las imágenes en color son de $481 \times 321$ y de $321 \times 481$. La comparación se realizó con algoritmos que modifican el histograma y otra que utiliza la transformada de top-hat multiescala. La valoración de los resultados experimentales se realizaron con métricas que evalúan el contraste local y global. El algoritmo propuesto obtuvo mejores valoraciones numéricas y visuales para todos los casos probados, tanto para imágenes en escala de grises e imágenes en color.
%\textcolor{Micolor1}{Los resultados experimentales muestran que el algoritmo propuesto obtuvo mejores resultados numéricos y visuales para todos los casos probados, tanto para imágenes en escala de grises e imágenes en color.} 
%\textcolor{Micolor2}{El algoritmo propuesto obtuvo mejores valoraciones numéricas y visuales para todos los casos probados, tanto para imágenes en escala de grises e imágenes en color.} 
\end{abstract}

