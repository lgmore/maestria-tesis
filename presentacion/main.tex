% Welcome! This is the unofficial University of Udine beamer template.

% See README.md for more informations about this template.

% This style has been developed following the "Manuale di Stile"
% (Style Manual) of the University of Udine. You can find the
% manual here: https://www.uniud.it/it/ateneo-uniud/ateneo-uniud/identita-visiva/manuali-immagine-stile/manuale-stile

% Note: for some reason, the RGB values specified in the manual
% do NOT render correctly in Beamer, so they have been redefined
% for this document using the high level chromo-optic deep neural 
% quantistic technology offered by Microsoft Paint's color picker.

% We defined four theme colors: UniBrown, UniBlue, UniGold
% and UniOrange. For example, to write some uniud-brownish
% text, just use: \textcolor{UniBrown}{Hello!}

% Note that [usenames,dvipsnames] is MANDATORY due to compatibility
% issues between tikz and xcolor packages.

\documentclass[usenames,dvipsnames]{beamer}
\usepackage[utf8]{inputenc}
\usepackage{verbatim}
\usepackage{adjustbox} % for \adjincludegraphics
\usepackage{mathrsfs}  
\usepackage{esvect}
\usetheme{uniud}

%%% Bibliography
\usepackage[style=authoryear,backend=biber]{biblatex}
\addbibresource{bibliography.bib}

%%% Suppress biblatex annoying warning
\usepackage{silence}
\usepackage[most]{tcolorbox}
\WarningFilter{biblatex}{Patching footnotes failed}

%%% Some useful commands
% pdf-friendly newline in links
\newcommand{\pdfnewline}{\texorpdfstring{\newline}{ }} 
% Fill the vertical space in a slide (to put text at the bottom)
\newcommand{\framefill}{\vskip0pt plus 1filll}


\title[Facultad Politécnica - UNA]{Contrast Enhancement of Color Images using a Multi-Objective Optimization Framework}
\date[Oct 25, 2017]{Oct 25, 2017}
\author[Luis G. Moré]{
  Luis G. Moré
  \pdfnewline
  \texttt{lmore@pol.una.py}
}
\institute{Facultad Politécnica - Universidad Nacional de Asunción}

\begin{document}

\begin{frame}
\titlepage
\end{frame}

\begin{frame}{Outline}
\tableofcontents
\end{frame}

\section{Introducción}
\begin{frame}{Introducción}

\begin{columns}[t]
\begin{column}{.3\textwidth}
\tcbox{\adjincludegraphics[width=0.85\linewidth,valign=t]{graphics/lena}}

\end{column}
\begin{column}{.7\textwidth}
\begin{itemize}
\item La mejora del contraste es un proceso de transformación de la imagen, con el objetivo de obtener una nueva imagen con un contraste más definido.
\item Se busca obtener imágenes más aptas para algún proceso posterior.
\item La mejora del contraste es un área de investigación atractiva en el procesamiento de imágenes.
\end{itemize}
% \hfill-- Carl Friedrich Gauss
\end{column}
\end{columns}


\end{frame}

\begin{frame}{Introducción}

\begin{columns}[t]
\begin{column}{.3\textwidth}
\tcbox{\adjincludegraphics[width=0.85\linewidth,valign=t]{graphics/lena}}

\end{column}
\begin{column}{.7\textwidth}
\begin{itemize}
\item Una técnica importante para la Mejora del Contraste es la Ecualización del Histograma.
\item Ésta técnica es directa y efectiva en el trabajo de Mejora del Contraste.
\item Existen enfoques globales y locales de Ecualización del Histograma.
\item Los enfoques locales son efectivos para el realce de detalles finos de la imagen digital.
\end{itemize}
% \hfill-- Carl Friedrich Gauss
\end{column}
\end{columns}


\end{frame}

\begin{frame}{Introducción}

\begin{columns}[t]
\begin{column}{.3\textwidth}
\tcbox{\adjincludegraphics[width=0.85\linewidth,valign=t]{graphics/lena}}

\end{column}
\begin{column}{.7\textwidth}
\begin{itemize}
\item En las imágenes digitales en escala de gris solamente es necesario considerar la información representada por los niveles de intensidad de los pixeles.
\item En las imágenes a color, es necesario además tener en cuenta la información de color representada, lo cual representa un problema adicional en el proceso.
\end{itemize}
% \hfill-- Carl Friedrich Gauss
\end{column}
\end{columns}


\end{frame}

\begin{frame}{Introducción}

\begin{columns}[t]
\begin{column}{.3\textwidth}
\tcbox{\adjincludegraphics[width=0.85\linewidth,valign=t]{graphics/lena}}

\end{column}
\begin{column}{.7\textwidth}
\begin{itemize}
\item En éste trabajo se busca atacar el problema de la Mejora del Contraste de imágenes digitales a color con un enfoque de Optimización Multi-Objetivo aplicado sobre un algoritmo de Ecualización del Histograma bien conocido.
\item Se busca obtener un balance entre el realce de detalles de la imagen digital y el mantenimiento de la información de brillo y de color.
\end{itemize}
% \hfill-- Carl Friedrich Gauss
\end{column}
\end{columns}


\end{frame}


% \begin{frame}[fragile]
% \frametitle{Compiling}

% \begin{alertblock}{Warning}
% You can ignore this slide if you're working with Overleaf.
% \end{alertblock}

% To compile this deck you'll need the \texttt{biber} package. Probably your \TeX editor already supports it; if not, you will easily find online the instructions to install it.

% \vskip 0.5cm

% If you're not using an editor, you can compile this presentation using the command line by running:

% \begin{verbatim}
% $ pdflatex main.tex
% $ biber main.bcf
% $ pdflatex main.tex
% $ pdflatex main.tex
% \end{verbatim}


% \end{frame}

\section{Marco Teórico}

\begin{frame}{Marco Teórico}

\begin{columns}[t]
\begin{column}{.3\textwidth}
\tcbox{\adjincludegraphics[width=0.85\linewidth,valign=t]{graphics/lena}}

\end{column}
\begin{column}{.7\textwidth}
\begin{itemize}
\item Es necesario revisar dos partes:
	\begin{itemize}
		\item El proceso de Ecualización del Histograma,
		\item La Meta-Heurística Aplicada.
	\end{itemize}
\end{itemize}
% \hfill-- Carl Friedrich Gauss
\end{column}
\end{columns}


\end{frame}

\begin{frame}{Marco Teórico}

\begin{columns}[t]
\begin{column}{.3\textwidth}
\tcbox{\adjincludegraphics[width=0.85\linewidth,valign=t]{graphics/lena}}

\end{column}
\begin{column}{.7\textwidth}
\begin{itemize}
\item Se realiza una separación de la información de intensidad de la información de color para el proceso de ecualización del histograma.
\item Se adoptaron representaciones de color bien conocidas para operar sobre la intensidad de la imagen digital:
	\begin{itemize}
		\item $RGB$ (\textit{Red, Green, Blue})
		\item $YCbCr$
	\end{itemize}
\end{itemize}
% \hfill-- Carl Friedrich Gauss
\end{column}
\end{columns}

\end{frame}


% \begin{frame}{Colors}

% For this template we defined four colors, following the Style Manual of the University of Udine:
% \begin{itemize}
% \item \textcolor{white}{\marker{\texttt{UniOrange}}}
% \item \textcolor{white}{\marker[UniBlue]{\texttt{UniBlue}}}
% \item \textcolor{white}{\marker[UniBrown]{\texttt{UniBrown}}}
% \item \textcolor{white}{\marker[UniGold]{\texttt{UniGold}}}
% \end{itemize}

% \vskip 0.5cm

% You can use these colors as you want in your presentation. For example, you can \textbf{\textcolor{UniGold}{color the text in gold}} by writing \texttt{\textbackslash\{UniGold\}\{my gold text\}}.

% \vskip 0.5cm

% We also redefined many of the most common \LaTeX{} and Beamer commands, like \texttt{itemize}, \texttt{block}, etc. You will see samples of these commands in the following slides.

% \end{frame}

%\section{Blocks}

\begin{frame} 
\frametitle{Marco Teórico} 
% \framesubtitle{And also some blocks.} 
\begin{exampleblock}{\textit{Red, Green, Blue (RGB)}}
Las imágenes digitales se representan inicialmente en $RGB$, por lo que se tiene un array de $N \times M \times 3$ pixeles de color.

Cada pixel se representa como un elemento $[z_r \quad z_g \quad z_b]$ donde $z_r,z_g,z_b$ son los componentes de rojo, verde y azul del pixel a color en una ubicación específica. Los componentes se mezclan para obtener el color representado por el pixel de la imagen.
\end{exampleblock} 

\centering
% \tcbox{\adjincludegraphics[width=0.25\linewidth,valign=t]{graphics/lena}}
\adjincludegraphics[width=0.30\linewidth,valign=t]{graphics/RGB_Cube_Show_lowgamma_cutout_b}

\end{frame}

\begin{frame} 
\frametitle{Marco Teórico} 
% \framesubtitle{And also some blocks.} 
\begin{exampleblock}{\textit{YCbCr}}
\textit{YCbCr} es un espacio de color definido a través de una transformación matemática de coordenadas, a partir de un espacio de color $RGB$ asociado.

La ventaja de ésta representación es que separa la información de intensidades de la imagen digital de la información de color presente.
\end{exampleblock}

\centering
% \tcbox{\adjincludegraphics[width=0.25\linewidth,valign=t]{graphics/lena}}
% \adjincludegraphics[width=0.30\linewidth,valign=t]{graphics/RGB_Cube_Show_lowgamma_cutout_b}

\end{frame}

\begin{frame} 
\frametitle{Marco Teórico} 
% \framesubtitle{And also some blocks.} 
\begin{exampleblock}{\textit{YCbCr}}
\textit{YCbCr} es un espacio de color definido a través de una transformación matemática de coordenadas, a partir de un espacio de color $RGB$ asociado.

Otra ventaja importante es que la conversión a partir de $RGB$, y luego de vuelta a $RGB$ es directa:
\end{exampleblock}
\begin{equation}
\begin{bmatrix}
    Y \\
    C_b \\
    C_r 
\end{bmatrix} =
 \begin{bmatrix}
    16  \\
    128 \\
    128
\end{bmatrix}
+
 \begin{bmatrix}
    65.481 & 128.553 & 24.966 \\
    -37.797 & -74.203 & 112.000 \\
    112.000 & -93.786 & -18.214 
\end{bmatrix}
\begin{bmatrix}
   R \\
   G \\
   B 
\end{bmatrix}
\end{equation}
\begin{equation}
\begin{bmatrix}
    R \\
    G \\
    B 
\end{bmatrix} =
 \begin{bmatrix}
    Y + 1.402 \cdot (C_r - 128) \\
    Y -0.34414 \cdot (C_b - 128) - 0.71414 \cdot (C_r - 128) \\
    Y + 1.772 \cdot  (C_b - 128) 
\end{bmatrix}
\end{equation}
\centering
% \tcbox{\adjincludegraphics[width=0.25\linewidth,valign=t]{graphics/lena}}
% \adjincludegraphics[width=0.30\linewidth,valign=t]{graphics/RGB_Cube_Show_lowgamma_cutout_b}

\end{frame}

\begin{frame} 
\frametitle{Marco Teórico} 
% \framesubtitle{And also some blocks.} 
\begin{exampleblock}{\textit{Contrast Limited Adaptive Histogram Equalization}}

Es un algoritmo de Mejora del Contraste bien conocido, diseñado para su aplicación en distintos tipos de imágenes. 

$CLAHE$ es una variación del algoritmo \textit{Adaptive Histogram Equalization}, el cual es un algoritmo de Mejora del Contraste local. 

\end{exampleblock}
\centering
\tcbox{\adjincludegraphics[width=0.55\linewidth,valign=t]{graphics/Clahe-redist.jpg}}

\centering
% \tcbox{\adjincludegraphics[width=0.25\linewidth,valign=t]{graphics/lena}}
% \adjincludegraphics[width=0.30\linewidth,valign=t]{graphics/RGB_Cube_Show_lowgamma_cutout_b}

\end{frame}

\begin{frame} 
\frametitle{Marco Teórico} 
% \framesubtitle{And also some blocks.} 
\begin{exampleblock}{\textit{Entropía de la Imagen}}

La Entropía de la imagen es una métrica que muestra la cantidad de información representada en la imagen digital.

La entropía de la imagen y su contraste están relacionados a la distribución de intensidad de las imágenes digitales, por lo que esta métrica es apta para medir variaciones del constraste como consecuencia de transformaciones aplicadas a la misma.

\end{exampleblock}
% \centering
% \tcbox{\adjincludegraphics[width=0.55\linewidth,valign=t]{graphics/Clahe-redist.jpg}}

\centering
% \tcbox{\adjincludegraphics[width=0.25\linewidth,valign=t]{graphics/lena}}
% \adjincludegraphics[width=0.30\linewidth,valign=t]{graphics/RGB_Cube_Show_lowgamma_cutout_b}

\end{frame}

\begin{frame} 
\frametitle{Marco Teórico} 
% \framesubtitle{And also some blocks.} 
\begin{exampleblock}{\textit{Entropía de la Imagen}}

La Entropía de la Imagen se define como se muestra abajo:

\end{exampleblock}

\begin{equation}
\mathscr{H}= -\sum_{i=0}^{n-1} p_i \text{log}_2(p_i) \qquad \mathscr{H} \in \{0,...,\text{log}_2(n)\}
\end{equation}

donde

\begin{equation}
p_i=\frac{c_i}{N}, \qquad \sum_{i=1}^n c_i = N, \qquad i= 1,2, ..., n,
\end{equation}

\end{frame}    

\begin{frame} 

\frametitle{Marco Teórico} 
% \framesubtitle{And also some blocks.} 
\begin{exampleblock}{\textit{Structural Similarity Index}}
Es una métrica bien conocida que mide atributos importantes de la imagen tales como la \textit{Luminancia, Contraste} y la \textit{Estructura}

El objetivo de $SSIM$ es el de medir la distorsión de la imagen como consecuencia del proceso de mejora del contraste.

Dadas una imagen de entrada $I_x$ y una de salida $T_y$ $SSIM$ se define como se muestra abajo:

\end{exampleblock}

\begin{equation}
SSIM(I,T) = \frac{(2\mu_{I_x} \mu_{T_y}+E_1)(2\sigma_{I_xT_y}+E_2)}{(\mu^2_{I_x}+\mu^2_{T_y}+E_1)(\sigma^2_{I_x} + \sigma^2_{T_y}+E_2)} \qquad SSIM \in [0,1]
\end{equation}

\end{frame}    

\begin{frame}
\frametitle{Marco Teórico} 
% \framesubtitle{And also some blocks.} 
\begin{exampleblock}{\textit{Multi-Objective Particle Swarm Optimization (MOPSO)}}

$MOPSO$ es una meta-heurística que emula el comportamiento social de las bandadas de pájaros.

Cada partícula $\vv{x}$ realiza una búsqueda dentro de un espacio $\Omega$, y para cada generación $t$, cada solución $\vv{x}$ se actualiza de acuerdo a:

\begin{equation}\label{eq:posicion1}
\vv{x}_i(t) = \vv{x}_i(t-1) + \vv{v}_i(t)
\end{equation}

Donde $\vv{v}$ se conoce como el factor de velocidad, y está dado por:

\begin{equation}\label{eq:velocidad1}
 \vv{v}_i(t) = w \cdot (t-1) + C_1 \cdot r_1 \cdot (\vv{x}_{p_i} - \vv{x}_i) + C_2 \cdot r_2 \cdot (\vv{x}_{g_i} - \vv{x_i})
\end{equation}

\end{exampleblock}

% \begin{equation}
% SSIM(I,T) = \frac{(2\mu_{I_x} \mu_{T_y}+E_1)(2\sigma_{I_xT_y}+E_2)}{(\mu^2_{I_x}+\mu^2_{T_y}+E_1)(\sigma^2_{I_x} + \sigma^2_{T_y}+E_2)} \qquad SSIM \in [0,1]
% \end{equation}

\end{frame}

\section{Formulación del Problema Planteado}

\begin{frame}
\frametitle{Formulación del Problema Planteado} 
% \framesubtitle{And also some blocks.} 
\begin{exampleblock}{\textit{Formulación del Problema Planteado}}

Dada una imagen a color $I$, con $M \times N$ pixeles, y un vector $\vv{x}= (\mathscr{R}_x, \mathscr{R}_y, \mathscr{C})$, donde $\mathscr{R}_x$ y $\mathscr{R}_y$ son regiones contextuales y $\mathscr{C}$ es el \textit{Clip Limit}, se busca un conjunto de soluciones no dominadas $\mathscr{X}$, que maximiza simultáneamente las funciones objetivo $f_1,f_2,f_3,f_4$:

\begin{equation}
f(I,\vv{x}) = [f_1(I,\vv{x}),f_2(I,\vv{x}),f_3(I,\vv{x}),f_4(I,\vv{x})]; \qquad f_1,f_2,f_3,f_4 \in [0,1]
\end{equation}


\end{exampleblock}

% \begin{equation}
% SSIM(I,T) = \frac{(2\mu_{I_x} \mu_{T_y}+E_1)(2\sigma_{I_xT_y}+E_2)}{(\mu^2_{I_x}+\mu^2_{T_y}+E_1)(\sigma^2_{I_x} + \sigma^2_{T_y}+E_2)} \qquad SSIM \in [0,1]
% \end{equation}

\end{frame}

% \subsection{Enumerates and itemizes}

% \begin{frame}{Enumerates and itemizes}

% This is an example of \texttt{itemize}.
% \begin{itemize}
% 	\item A long time ago in a galaxy far, far away...
% \end{itemize}
% And this is an example of \texttt{enumerate}.

% \begin{enumerate} 
%   \item Go to the Death Star.
%   \item Find the exhaust port.
%   \item Make the perfect shot.
%   \item Become an hero.
% \end{enumerate}
% \end{frame}

% \subsection{Description}

% \begin{frame}[fragile]
% \frametitle{Description}
% This is an example of \texttt{description}.

% \begin{description}
% \item<2->[Vader] \emph{I am} your father.
% \item<1->[Luke] No. No! That's not true! \textbf{That's impossible!}
% \end{description}

% \begin{uncoverenv}<3>
%   \vskip 0.5cm
%   And while we're here, let's have a look to \texttt{verbatim} as well, to see how we made items appear in arbitrary order:
%   \vskip 0.5cm
%   \begin{verbatim}
% \begin{description}
%   \item<2->[This is the first item] one
%   \item<1->[This is the second item] two
% \end{description}
%   \end{verbatim}
% \end{uncoverenv}

% \end{frame}

\section{Propuesta}

\begin{frame}
\frametitle{Propuesta - CMOPSO-CLAHE} 
% \framesubtitle{And also some blocks.} 
\begin{exampleblock}{\textit{Diagrama esquemático de CMOPSO - CLAHE}}

\centering
\tcbox{\adjincludegraphics[width=0.75\linewidth,valign=t]{graphics/ccis2016.png}}

\end{exampleblock}

\end{frame}

% \begin{frame}{Maths}
% A formula will look like this: 
% \begin{center}
%  $x^2 + y^2 = z^2$
% \end{center}

% You can number equations as well:
% \begin{equation}
% 1+1=2
% \end{equation}

% \begin{equation}
% 1+1=2 \tag{custom label!}
% \end{equation}

% \vskip 0.5cm

% If you want to use the default \LaTeX{} math fonts, just go to \texttt{beamerfontthemeuniud.sty} and uncomment the line containing `\texttt{\textbackslash usefonttheme[onlymath]\{serif\}}'.

% \end{frame}

% \begin{frame}{Theorems}

% The usual \texttt{theorem}, \texttt{corollary}, \texttt{definition}, \texttt{definitions}, \texttt{fact}, \texttt{example} and \texttt{examples} blocks are available as well.

% \begin{theorem}
% There exists an infinite set.
% \end{theorem}
% \begin{proof}
% This follows from the axiom of infinity.
% \end{proof}
% \begin{example}[Natural Numbers]
% The set of natural numbers is infinite.
% \end{example}

% \end{frame}

\section{Resultados y discusión}

\begin{frame}
\frametitle{Resultados y discusión} 
% \framesubtitle{And also some blocks.} 
\begin{exampleblock}{\textit{Parámetros Iniciales }}

\end{exampleblock}

\begin{table}[H]
\setlength{\abovecaptionskip}{2pt plus 3pt minus 2pt} % Chosen fairly arbitrarily
\caption[Parámetros de entrada para $MOPSO$]{Parámetros iniciales para $CMOPSO-CLAHE$}
\begin{center}
 \begin{tabular}{||c c | c c||} 
 \hline
 Parámetro & Valor & Parámetro & Valor \\ [0.5ex] 
 \hline\hline
 $lower\_limit_{\mathscr{R}_x}$ & $2$ & $upper\_limit_{\mathscr{R}_x}$ & $M/2$ \\ 
 \hline
 $lower\_limit_{\mathscr{R}_y}$ & $2$ & $upper\_limit_{\mathscr{R}_y}$ & $N/2$ \\  
 \hline
 $lower\_limit_{{\mathscr{C}}}$ & $0$ & $upper\_limit_{{\mathscr{C}}}$ & 0.5 \\
\hline
$\Omega$ & $100$ & $t_{max}$ & $100$ \\ 
\hline
$c_1$ $min$ & 1.5 & $c_1$ $max$ & 2.5 \\ 
\hline
$c_2$ $min$ & 1.5 & $c_2$ $max$ & 2.5 \\ 
\hline
$r_1$ $min$ & 0.0 & $r_1$ $max$ & 1.0 \\ 
\hline
$r_2$ $min$ & 0.0 & $r_2$ $max$ & 1.0 \\
\hline
\end{tabular}
\end{center}
\label{table:parametrospso}
\end{table}

\end{frame}



% \begin{frame}{Other blocks}

% Here we display examples of \texttt{abstract}, \texttt{verse}, \texttt{quotation}, and \texttt{quote}.

% \vskip 0.5cm

% \begin{abstract}
% This is an abstract.
% \end{abstract}
% \begin{verse}
% This is a verse.
% \end{verse}
% \begin{quotation}
% This is a quotation.

% \raggedleft -Han Solo
% \end{quotation}
% \begin{quote}
% A quote this is.

% \raggedleft -Yoda
% \end{quote}

% \end{frame}

\section{Conclusions}
% \begin{frame}[fragile]
% \frametitle{Bibliography}

% You can cite an article
% \begin{itemize}
% \item normally using \texttt{\textbackslash cite}, e.g.: (\cite{article1})
% \item or display the full citation using \texttt{\textbackslash fullcite}, e.g.:  \fullcite{article1}
% \end{itemize}

% \vskip 0.5cm
% Look at the code of the following slide to see how to automatically split the bibliography on many slides. You can also use \texttt{\textbackslash nocite\{*\}} to display the non-cited publications as well.

% \end{frame}

% \begin{frame}[t,allowframebreaks]
% \frametitle{Bibliography}

% \nocite{*} % will display the non-cited publications as well. Useful for a publication list.

% \printbibliography

% \end{frame}

%\section{Bonus Commands}

% \begin{frame}[fragile]
% \frametitle{Framecard}

% You can display a frame with a colored background and a huge text in the center using the command \texttt{\textbackslash framecard}.
% \vskip 0.5cm 
% For example, you can write:
% \begin{verbatim}
% \framecard{A SECTION\\TITLE}
% \end{verbatim}

% This will display a frame with a orange background and the phrase "A SECTION TITTLE" in the center. You can also use a custom color with \texttt{\textbackslash framecard}:
% \begin{verbatim}
% \framecard{A SECTION\\TITLE}
% \framecard[UniBlue]{A SECTION TITLE\\
% WITH A CUSTOM COLOR}
% \end{verbatim}
% You can see the results of the commands above in the following slides.

% \end{frame}

% \framecard{A SECTION\\TITLE}
% \framecard[UniBlue]{A SECTION TITLE\\WITH A CUSTOM COLOR}

% \begin{frame}[fragile]
% \frametitle{Framepic}

% You can display a frame with a background image using the command \texttt{\textbackslash framepic}. The image will be \textbf{adapted vertically} to fit the the frame. 

% For example, you can write:
% \begin{verbatim}
% \framepic{graphics/darth}{
% 	\framefill
%     \textcolor{white}{Luke,\\I am your supervisor}
%     \vskip 0.5cm
% }
% \end{verbatim}

% Alternatively, to make the background 50\% transparent, you can write \texttt{\textbackslash framepic[0.5]\{graphics/darth\}...}


% You can see the results of the commands above in the following slides.

% \end{frame}


% \framepic{graphics/darth}{
% 	\framefill
%     \textcolor{white}{Luke,\\I am your supervisor}
%     \vskip 0.5cm
% }

% \framepic[0.5]{graphics/darth}{
% 	\vfill
%     \begin{flushright}
%     \textcolor{red}{\textbf{Right-aligned text with\\Semi-transparent background}}
%     \end{flushright}	
% }

% \begin{frame}[t,fragile,allowframebreaks]
% \frametitle{Other bonus commands}

% We provide two other bonus commands:
% \begin{description}
% \item[\texttt{pdfnewline}] you can use \texttt{\textbackslash pdfnewline} to avoid the annoying \texttt{hyperref} related warnings when using newlines in the document's title, author, etc. For example, in this presentation the author is defined as:
% \begin{verbatim}
% \author[Luke Skywalker]{
%   Luke Skywalker, Ph.D.
%   \pdfnewline
%   \texttt{luke.skywalker@uniud.it}
% }
% \end{verbatim}
% \item[\texttt{marker}] you can use \texttt{\textbackslash marker} to highlight some text. The default color is \marker{orange}, but you can also \marker[UniBlue]{use a custom color}. For example:
% \begin{verbatim}
% \marker{Default color}
% \marker[UniBlue]{Custom Color}
% \end{verbatim}
% \item[\texttt{framefill}] you can use \texttt{\textbackslash framefill} to put the text at the bottom of a slide by filling all the vertical space.
% \end{description}

% \end{frame}

\end{document}